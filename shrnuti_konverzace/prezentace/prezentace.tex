\documentclass{beamer}
\usetheme{Madrid}
\usecolortheme{default}

\usepackage[utf8]{inputenc}
\usepackage[T1]{fontenc}
\usepackage[czech]{babel}
\usepackage{csquotes}
\usepackage[backend=biber,style=numeric]{biblatex}

\addbibresource{references.bib}

\title{Architektura softwarových systémů}
\subtitle{Shrnutí konverzace}
\author{AI Assistant}
\date{\today}

\begin{document}

\begin{frame}
    \titlepage
\end{frame}

\begin{frame}{Obsah}
    \tableofcontents
\end{frame}

\begin{frame}{Vizualizace}
    \begin{figure}
        \centering
        \includegraphics[width=0.8\textwidth]{../related_image.png}
        \caption{Vizualizace z konverzace}
    \end{figure}
\end{frame}

\section{Úvod do architektury SW systémů}
\begin{frame}{Úvod do architektury SW systémů}
    \begin{itemize}
        \item Softwarová architektura popisuje \textbf{vysokou strukturu systému}, komponenty a jejich interakce.
        \item Cílem je \textbf{zajištění kvality, flexibility a udržovatelnosti} systému.
        \item Klíčové role architektury: plánování, dokumentace, rozhodování.
    \end{itemize}
    \vspace{1cm}
    \footnotesize{Zdroj: \cite{sei_arch}}
\end{frame}

\section{Architektonické styly}
\begin{frame}{Architektonické styly - Přehled}
    \begin{itemize}
        \item \textbf{Vrstvená (Layered)} – logické vrstvy, separace zodpovědností.
        \item \textbf{Client–Server} – centralizovaná logika, komunikace přes síť.
        \item \textbf{Monolit} – jeden celek, jednoduchý na začátku, horší škálovatelnost.
        \item \textbf{Microservices} – malé nezávislé služby, škálovatelnost.
        \item \textbf{SOA} – služby s Enterprise Service Bus.
        \item \textbf{Event‑Driven} – komunikace přes události, real‑time systémy.
        \item \textbf{Hexagonální / Clean Architecture} – oddělení jádra od adaptérů.
    \end{itemize}
    \vspace{0.5cm}
    \footnotesize{Zdroj: \cite{arch_guild}}
\end{frame}

\begin{frame}{Monolit}
    \begin{figure}
        \centering
        \includegraphics[width=0.8\textwidth]{../../img/monolith-gemini.png}
        \caption{Monolitický styl}
    \end{figure}
\end{frame}

\begin{frame}{Client-Server}
    \begin{figure}
        \centering
        \includegraphics[width=0.8\textwidth]{../../img/client-server-chatgpt.png}
        \caption{Client-Server architektura}
    \end{figure}
\end{frame}

\begin{frame}{Microservices}
    \begin{figure}
        \centering
        \includegraphics[width=0.8\textwidth]{../../img/microservices-chatgpt.png}
        \caption{Microservices architektura}
    \end{figure}
\end{frame}

\section{Pohledy na SW architekturu}
\begin{frame}{Pohledy na SW architekturu}
    \begin{itemize}
        \item Pohledy umožňují různým stakeholderům \textbf{vidět systém podle svých potřeb}.
        \item \textbf{4+1 Views} (Philippe Kruchten):
        \begin{enumerate}
            \item Logický – co systém dělá
            \item Procesní – chování za běhu
            \item Vývojový – modulární struktura kódu
            \item Fyzický – nasazení a infrastruktura
            \item Scénáře / Use Cases – případy použití
        \end{enumerate}
        \item Alternativy: C4 model, ISO/IEC 42010.
    \end{itemize}
    \vspace{0.5cm}
    \footnotesize{Zdroj: \cite{wiki_4plus1}}
\end{frame}

\section{Modelování a dokumentace}
\begin{frame}{Modelování a dokumentace SW architektury}
    \begin{itemize}
        \item \textbf{Modelování} – vytváření diagramů a modelů pro pochopení systému.
        \item \textbf{Dokumentace} – soubor artefaktů: struktura, rozhodnutí, pohledy.
        \item Standardy: UML, C4, ArchiMate.
        \item \textbf{ADR} (Architecture Decision Records) – zaznamenání důležitých rozhodnutí.
        \item Doporučení: udržovat dokumentaci živou, používat nástroje (PlantUML, Structurizr, Archi).
    \end{itemize}
    \vspace{0.5cm}
    \footnotesize{Zdroj: \cite{soft_sys_design}}
\end{frame}

\section{Kvalitativní atributy}
\begin{frame}{Kvalitativní atributy SW architektury}
    \begin{itemize}
        \item \textbf{Klíčové atributy:} Dostupnost, Modifikovatelnost, Výkonnost, Bezpečnost, Integrovatelnost, Znovupoužitelnost, Testovatelnost, Uživatelská přívětivost.
        \item Slouží k definici \textit{nefunkčních požadavků} a ovlivňují architektonická rozhodnutí.
    \end{itemize}
    \vspace{1cm}
    \footnotesize{Zdroj: \cite{wiki_furps}}
\end{frame}

\section{Vybrané architektonické vzory}
\begin{frame}{Vybrané architektonické vzory}
    \begin{itemize}
        \item \textbf{Příklady:} Vrstvená, Client–Server, Event‑Driven, Microservices, Hexagonální, Microkernel, Space-Based, CQRS, P2P, Strangler, MVC.
        \item Vzory řeší opakující se architektonické problémy a podporují škálovatelnost, modularitu a udržovatelnost.
    \end{itemize}
    \vspace{1cm}
    \footnotesize{Zdroj: \cite{geeks_patterns}}
\end{frame}

\section{Datová architektura}
\begin{frame}{Datová architektura}
    \begin{itemize}
        \item Popisuje \textbf{strukturu, toky, úložiště a správu dat} napříč systémy a organizací.
        \item \textbf{Klíčové komponenty:} datové modely (konceptuální, logické, fyzické), datové toky, úložiště, governance, přístup.
        \item Důležitá pro: kvalitu, bezpečnost, analytiku, AI/ML, integraci a rozhodování.
    \end{itemize}
    \vspace{1cm}
    \footnotesize{Zdroj: \cite{sap_data_arch}}
\end{frame}

\section{Zdroje}
\begin{frame}[allowframebreaks]{Zdroje}
    \printbibliography
\end{frame}

\end{document}
