\documentclass{beamer}
\usetheme{Madrid}
\usecolortheme{default}

\usepackage[czech]{babel}
\usepackage[utf8]{inputenc}
\usepackage[T1]{fontenc}
\usepackage{hyperref}

% Meta data
\title{Architektura Softwarových Systémů}
\subtitle{Shrnutí klíčových konceptů}
\author{GitHub Copilot}
\date{\today}

% Bibliography setup
\usepackage[backend=biber,style=authoryear]{biblatex}
\addbibresource{references.bib}

\begin{document}

% Title slide
\begin{frame}
    \titlepage
\end{frame}

% Table of contents
\begin{frame}{Obsah}
    \tableofcontents
\end{frame}

% Related Image
\begin{frame}{Vizualizace}
    \begin{figure}
        \centering
        \includegraphics[width=0.8\textwidth]{../related_image.png}
        \caption{Vizualizace z konverzace}
    \end{figure}
\end{frame}

% Section 1: Úvod
\section{Úvod do architektur SW systémů}
\begin{frame}{Úvod do architektur SW systémů}
    \begin{itemize}
        \item \textbf{SW architektura} popisuje strukturu systému, komponenty a vztahy \parencite{bass2012software}.
        \item Slouží k řízení složitosti a zajištění kvalitativních atributů (výkonnost, bezpečnost, modifikovatelnost).
        \item \textbf{Hlavní cíle:}
        \begin{itemize}
            \item Komunikace mezi týmy
            \item Základ pro rozhodnutí o návrhu
            \item Dokumentace a údržba
        \end{itemize}
        \item Tvoří se na vysoké úrovni abstrakce, bez implementačních detailů \parencite{shaw1996software}.
    \end{itemize}
\end{frame}

% Section 2: Architektonické styly
\section{Architektonické styly}
\begin{frame}{Architektonické styly - Přehled}
    \begin{itemize}
        \item \textbf{Monolitický styl:} Vše v jedné aplikaci. Jednoduchý, ale hůře škálovatelný.
        \item \textbf{Klient-server:} Oddělení UI a logiky.
        \item \textbf{Vrstevnatá architektura:} Rozdělení do vrstev (prezentace, logika, data).
        \item \textbf{Pipes \& Filters:} Transformace datových toků.
        \item \textbf{Event-driven:} Reakce na události, asynchronní systémy \parencite{richards2020fundamentals}.
        \item \textbf{Microservices:} Nezávislé služby, komunikace přes API \parencite{fowler2014microservices}.
    \end{itemize}
\end{frame}

\begin{frame}{Monolitický styl}
    \begin{figure}
        \centering
        \includegraphics[width=0.8\textwidth]{../../img/monolith-gemini.png}
        \caption{Monolitický styl (Gemini)}
    \end{figure}
\end{frame}

\begin{frame}{Klient-Server}
    \begin{figure}
        \centering
        \includegraphics[width=0.8\textwidth]{../../img/client-server-chatgpt.png}
        \caption{Klient-Server (ChatGPT)}
    \end{figure}
\end{frame}

\begin{frame}{Microservices}
    \begin{figure}
        \centering
        \includegraphics[width=0.8\textwidth]{../../img/microservices-chatgpt.png}
        \caption{Microservices (ChatGPT)}
    \end{figure}
\end{frame}

% Section 3: Pohledy na SW architekturu
\section{Pohledy na SW architekturu}
\begin{frame}{Pohledy na SW architekturu}
    Různé perspektivy pro různé stakeholdery:
    \begin{itemize}
        \item \textbf{Logický pohled:} Hlavní komponenty a interakce.
        \item \textbf{Fyzický pohled:} Rozmístění SW na HW (deployment).
        \item \textbf{Procesní pohled:} Dynamické chování a toky dat.
        \item \textbf{Datový pohled:} Struktura a tok dat.
        \item \textbf{Vývojový pohled:} Modulární struktura pro vývoj.
    \end{itemize}
\end{frame}

% Section 4: Modelování a dokumentace
\section{Modelování a dokumentace}
\begin{frame}{Modelování a dokumentace SW architektury}
    \begin{itemize}
        \item \textbf{Cíle:} Sdílení znalostí, plánování, údržba.
        \item \textbf{Standardy a notace:}
        \begin{itemize}
            \item UML (Use Case, Class, Component, Deployment diagramy)
            \item ArchiMate
            \item C4 model \parencite{c4model}
        \end{itemize}
        \item \textbf{Klíčové diagramy:} Komponentní, Nasazení, Toků dat.
    \end{itemize}
\end{frame}

% Section 5: Kvalitativní atributy
\section{Kvalitativní atributy}
\begin{frame}{Kvalitativní atributy (Quality Attributes)}
    Vlastnosti systému nad rámec funkčnosti \parencite{bass2012software}:
    \begin{itemize}
        \item \textbf{Dostupnost (Availability):} Odolnost vůči chybám.
        \item \textbf{Modifikovatelnost (Modifiability):} Snadná změna.
        \item \textbf{Výkonnost (Performance):} Rychlost a efektivita.
        \item \textbf{Bezpečnost (Security):} Ochrana dat a přístupu.
        \item \textbf{Integrovatelnost, Znovupoužitelnost, Testovatelnost.}
        \item \textbf{Uživatelská přívětivost (Usability).}
    \end{itemize}
\end{frame}

% Section 6: Architektonické vzory
\section{Architektonické vzory}
\begin{frame}{Vybrané architektonické vzory}
    Ověřená řešení opakujících se problémů:
    \begin{itemize}
        \item \textbf{MVC (Model-View-Controller):} Oddělení dat, UI a logiky.
        \item \textbf{Layered (Vrstevnatý):} Hierarchické uspořádání zodpovědností.
        \item \textbf{Repository:} Centrální úložiště dat.
        \item \textbf{Broker:} Middleware pro komunikaci.
        \item \textbf{Peer-to-Peer:} Decentralizovaná spolupráce uzlů.
    \end{itemize}
\end{frame}

% Section 7: Datová architektura
\section{Datová architektura}
\begin{frame}{Datová architektura}
    Správa a tok dat v systému:
    \begin{itemize}
        \item \textbf{Datové modely:} ER diagramy, relační vs. NoSQL.
        \item \textbf{Úložiště:} Data warehouses, Data lakes.
        \item \textbf{Integrace:} Datové toky.
        \item \textbf{Správa:} Security, zálohování, kvalita dat.
    \end{itemize}
\end{frame}

% References
\section{Zdroje}
\begin{frame}[allowframebreaks]{Zdroje}
    \printbibliography
\end{frame}

\end{document}
